\documentclass{book}
\usepackage[utf8]{inputenc}
\usepackage[paperwidth=5.5in, paperheight=8.5in, top=0.75in, bottom=0.75in, outer=0.50in, inner=0.75in]{geometry}
\usepackage{setspace}
\usepackage{graphicx}
\usepackage{csquotes}
\usepackage{fancyhdr}
\usepackage{changepage}
\graphicspath{ {/home/spenteco/1/pg_fastText/novel_process/tex_templates/} }
\renewcommand*\footnoterule{}
\setlength{\parindent}{2em}
\setlength{\parskip}{0.5em}
\begin{document}
\pagenumbering{gobble}
{\fontsize{10pt}{12pt}\fontfamily{ptm}\selectfont
\newpage
\null

\newpage
\begin{center}
\vspace*{5mm}
{\Huge
\textsc{Descriptive Catalog}
\linebreak
\linebreak
\textsc{and}
\linebreak
\linebreak
\textsc{Bibliography}
\linebreak
\linebreak
\textsc{for the}
\linebreak
\linebreak
\textsc{MoMoGenMo Treasury}
\linebreak
\linebreak
\textsc{of}
\linebreak
\linebreak
\textsc{n-plus Literature}
\linebreak
\linebreak
}
{\Large
\linebreak
\linebreak
\linebreak
\linebreak
\linebreak
\linebreak
by
\linebreak
\linebreak
\linebreak
}
{\huge
Stephen M. Pentecost
}
\linebreak
\linebreak
\linebreak
\linebreak
\linebreak
\linebreak
\linebreak
\linebreak
\linebreak
{\LARGE 
\texttt{robineggsky.com}
}
\linebreak
\linebreak
{\Large 
\textsc{St. Louis, 2016}
}
\end{center}

\newpage
\vspace*{1.0in}
Copyright 2016 Stephen M. Pentecost.
\bigbreak
\bigbreak
\bigbreak
\par
This work is licensed under the Creative Commons Attribution-NonCommercial-ShareAlike 4.0 International License. To view a copy of this license, visit http://creativecommons.org/licenses/by-nc-sa/4.0/.
\bigbreak
\par
The Descriptive Catalog was created by algorithmically transforming the copy of \textit{The Reader's Digest of Books} by Helen Rex Keller available at the Internet Archive website.  The Bibliography was created by transforming selections from the Project Gutenberg metadata.
\bigbreak
\par
For more information, see
\par
\vspace*{0.10in}
\texttt{http://robineggsky.com/posts/momogenno.html}.
\par

\newpage
\null

\newpage
\null

\newpage
\pagenumbering{arabic}
\pagestyle{fancy}
\fancyhf{}
\newpage
\begin{center}
\textbf{\textsc{Descriptive Catalog}}
\end{center}
\fancyhead[RO]{\textsc{Descriptive Catalog}}
\fancyhead[LE]{\textsc{Stephen M. Pentecost}}
\fancyfoot[RE]{\textsc{The MoMoGenMo Treasury of n-plus Literature}}
\fancyfoot[LO]{\texttt{http://robineggsky.com}}
\fancyfoot[LE,RO]{\thepage}

\par
\textsc{Task}, \textsc{The}, a descriptive and reflective poem by William Cowper, published in 1785. It was begun at the instance of the poet's friend, Georgina Austen, who playfully asked him to write a poem in blank-verse about a sofa. Accepting the challenge Cowper traces in about one hundred Miltonic lines the evolution of the sofa from the chair. He then proceeds discursively to enlarge on the pleasures of country walks, the delights of gardening, and the coziness of the winter fireside, mingling these descriptive passages with autobiographic records of religious experience, satirical attacks on the luxury of cities, the corruption of politicians and the worldliness of the clergy, pietistic denunciations of deism, skepticism, and natural science, and outbursts of humanitarian sympathy for slaves, blind animals, and all who are oppressed. These and other topics occupy six books entitled respectively: 'The Sofa,' 'The Time-Piece' (i.e., the omens of future judgment), 'The Garden,' 'The Winter Dinner,' 'The Winter Morning Walk,' 'The Winter Walk at Noon.' 'The Task' reflects the enthusiasm for natural scenery, the impulse to self-revelation, and the eagerness to relieve suffering, of the later eighteenth century. The novelist is a refined, sensitive Christian countryman with a gift of easy, graceful expression, and a nature of little sensibility and quiet humor. The neurotic strain which so sadly affected his peace and happiness left no trace on this poem.
\vspace{0.25in}
\fancyhead[RO]{\textsc{Bibliography}}
\fancyhead[LE]{\textsc{Stephen M. Pentecost}}
\fancyfoot[RE]{\textsc{The MoMoGenNo Treasury of n-plus Literature}}
\fancyfoot[LO]{\texttt{http://robineggsky.com}}
\fancyfoot[LE,RO]{\thepage}
\newpage
\begin{center}
\textbf{\textsc{Bibliography}}
\end{center}
\begin{flushleft}
\parindent=0pt
\hangindent=10mm
\par
Cowper, William (1731-1800).  \textit{Task, and Other Poems, The}.  From Cowper, William (1731-1800), \textit{The Task, and Other Poems} (982.txt).\linebreak
\vspace{-5mm}
\end{flushleft}

}
\end{document}
