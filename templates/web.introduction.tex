## -*- coding: utf-8 -*-
\newpage
\pagenumbering{arabic}
\pagestyle{fancy}
\fancyhf{}
\fancyhead[RO]{Introduction}
\fancyhead[LE]{Stephen M. Pentecost}
\fancyfoot[RE]{The MoMoGenMo Treasury of n-plus Literature}
\fancyfoot[LO]{http://robineggsky.com}
\fancyfoot[LE,RO]{\thepage}
\begin{center}
\par
INTRODUCTION
\par
A DARK AND STORMY NIGHT – ALARUMS – SHADES OF YELLOW – THE BATTLE OF THE PICNIC TABLE – THE BEAST RETREATS
\end{center}
\par
It was a dark and stormy night; the wind blew broken clouds in gusts across the face of the full moon, its light moving like a discotheque strobe across my camp (for it is in the northeastern Ozarks that our scene lies), rattling bare tree limbs across each other, and fiercely agitating the scanty flame of my campfire, struggling to light the little circle containing my tent, the picnic table, and my lawn chair.  But no matter, for  quite fatigued from the day's exertions, I was not long for the warmth of my bedroll, and for the reward of a well-deserved slumber.  My chin drooped to my chest, and I began to nod off.
\par
I might have fallen asleep there, in my chair, and comfortably so, but I was soon awakened by the most horrid and hideous shriek from the dark woods, and by the sound of some very large animal crashing through the underbrush.  I leapt to my feet, instantly awake, and began to shout and wave my arms in a threatening manner, attempting to scare away whatever lurked there, beyond, in the dark.
\par
But, alas, my demonstrations proved ineffective.  The shrieks and, as the animal drew closer, its roars grew louder, its crashing in brush more urgent.  A shape, unlike any I had seen before, appeared at the edge of my campfire's pitiful glimmer.  I fumbled for my flashlight, but, alas! in the confusion I must have kicked it into the shadows!  How I fumbled for my flashlight, but, oh! how futilely I fumbled!
\par
When I looked up, the beast had stepped into the light, and how I froze in terror at the vision!  It was a sight such as I have never seen before, the beast, no, the brute, towered over my camp, its large and rounded black and white form casting enormous shadows on the trees behind it.  And its feet, its two large yellow feet, how easily they might stamp me flat!
\par
For a moment, as I pondered the color of the beast's two large feet (yellow?  or tan, but cast in a warmer hue by the campfire?  corn-silk, perhaps that was the color?  Or goldenrod?), the beast was as shocked into stillness as I was.  Perhaps it found my costume (fur hat, red flannel shirt, Wranglers, and a pair of very large boots) as shocking as I found its appearance!\footnote{Many readers may find in this narrative an allegory touching upon recent events in our country, and especially, given this narrative's preoccupation with large feet (``we all know what that means"), will hear a reference to an exchange in one of our Presidential Debates.  Nothing could be further from my intention, not that I (here, the reader should imagine a certain kind of hand waving), ``have anything to worry about in that department, believe you me."}
\par
The beast broke the spell, and shambled toward the picnic table, where were laid out my kitchen and my books.  I retreated to see what it would do, and to gather my courage, should I find need for it.
\par
The beast began to poke around amid the equipage arrayed on the picnic table.  Instantly, the heat of courage and fortitude coursed through my veins.  Of course, I surmised, it's after my breakfast!  Without a thought, and with a shout worthy of Achilles, I launched myself at the beast, and struck it a series of blows around its buttocks and, leaping high in righteous anger, about its lower back.
\par
``UTHER UTTER!" the beast cried, and dealt me a backhanded blow, knocking me across the campsite.  As I pulled myself to my knees, I noticed that the beast was picking through my books, and not rooting through the very big and very red ice chest which contained my breakfast.  Could it be that the beast was as new to this sort of thing as I was, and that it didn't realize that my breakfast was not camouflaged beneath a pile of books, but was instead concealed within the very large and very red ice chest?  ``PFIT," the beast snorted, and flung a C.J. Box novel over its shoulder.  ``PFIT", it cried again, flinging away Stephen King's latest.
\par
Still, my blood was up, and as long as the beast was at the picnic table, there was a chance that it might look inside the ice chest, inspect its contents, and deprive me of my breakfast.  Steeling my resolve, I launched myself at the beast again, this time fetching it a number of blows with my lawn chain.  ``UTHER UTTER!" beast shrieked again, and again it knocked me aside.  Again, I staggered up, only to see the beast digging through my books in search of my breakfast, discarding the books one by one.  By now, it had worked its way down to my collection of poetry.  ``PFIT," it yowled, discading a volume by Mary Oliver.  ``MO PFIT," it cried, tossing aside the collected verses of Billy Collins.
\par
Tiresome it would be to tell how long the battle continued, how many times I picked myself up and set upon the beast again, what common campsite objects I used as weapons in the battle,\footnote{Besides the lawn chair (which was really one of those new fangled collapsing nylon chairs, and not the sort of 50's contraption that ``lawn chair" brings to mind): a stave of firewood; a big rock; a lantern, unfortunately not lit; a small, portable barbeque grill, charcoal still warm; another rock, smaller than the first; one of my very large boots; a dutch oven containing an uneaten portion of persimmon cobbler, the cobbler weighing as much as the oven; a pillow (by this time, I had been knocked about the head); more firewood, ranging in size from kindling to yule log; my other very large boot; etc.} how often the beast swatted me aside, or even how much more difficult it became each time to rise again and renew the defense of my breakfast.  The pile of my books grew smaller and smaller.  The campfire burned down.  The moon set.
\par
Finally, after what seemed like hours, as I drew myself up for one last assault, I noticed that the beast was gone, leaving behind only the wreckage of my camp and the sound of it crashing through the woods.  Exhausted, I collapsed onto my bedroll and passed immediately into the sleep of the dead.
\par
\begin{center}
AN EARLY MORNING – A HEALTHY AND HEARTY BREAKFAST – THE WRECKAGE OF THE CAMPSITE – A MISSING BOOK
\end{center}
\par
Morning found me face down in my bedroll, hair askew and bootless, groggy, and famished from the night's exertions.  Sustenance!  My fatigued limbs demanded sustenance!  Fortunately, I saw untouched on the picnic table my very large and very red ice chest, so my first thought—nay, my first act—was to prepare and consume a healthy and hearty breakfast of fried ham and hominy grits; the life of the Missoura literati is one of constant tribulation and trouble, but it is not without its compensations.
\par
Thus fortified in the manner to which I am accustomed, and after locating my very large boots, I set about to survey the wreckage of my camp.  Books were scattered everywhere, as were the various bits of the equipment which I had used in defending my breakfast.  All was quickly put right, and all seemed more or less as it should be, although something seemed amiss.  The lawn chair?  No, it was easily bent straight.  The rocks?  They were broken into small pieces, but there were more where they came from.  The firewood?  Splintered, but it would burn better that way.  Something was wrong, but I couldn't quite put my finger on it.
\par
Finally, it dawned on me, as slowly as the sun rising over my camp.  My copy of the Georgics was missing.  I searched again, and searched until there was no other conclusion: the beast had made off with my copy of the Georgics.\footnote{Dryden's translation, which I prefer to the original in much the same way and for the same reasons that certain fundamentalists prefer the King James Version.}  It was befuddling in the extreme: why would the beast have made off with Dryden, and left behind the perfectly perfect ham and hominy grits?  And what was this gummy red material in the dirt around the picnic table, and what were these strange fibers on the ground, neither exactly fibers of fur nor precisely fibers of feather?  Panicked, I checked myself for wounds, but I could find none.  And I no longer have enough hair to account for the fibrous material, although had I had hair, it would have been of the same colors of white and black.  Awareness came slowly again, as it often does as I grow older: in the course of our battle, had I wounded the beast who had stolen my Dryden?
\newpage
\par
\begin{center}
A BOY SCOUT'S WOODSMAN'S SKILLS – CHANGING FASHIONS IN OUTDOOR EQUIPMENT – TRACKING THE BEAST – BYPASSERS OFFER DIRECTIONS – ENTERING ITS LAIR – THE BEAST EXPIRES
\end{center}
\par
Fortunately, the Boy Scouts of my youth, so unlike the enervated and pantywaisted Boy Scouts of today, taught the tracking and recovery of wounded game,\footnote{I seem to recall that the techniques were covered in the test for the Big Game Hunting merit badge, which I did earn.  Unfortunately, I never had the opportunity to put such techniques into practice, since I shot poorly at the rifle range, and consequently never earned the Marksmanship merit badge.  And, although I take full responsibility for my failure to breath and squeeze properly on the rifle range, and thus for my failure to the earn the Marksmanship merit badge, I still have a profound difference of opinion with the Boy Scouts on the subject: the beast was literally as big as the side of a barn.  Not as big as the side of working dairy barn, but easily as big as the side of a one cow, ten chicken barn of the sort my grandfather owned, and at the hitting of which I proved proficient at an early age, since I used it as a backstop for whatever target I set up for my BB gun.  To me, in retrospect it seems that if the Boy Scouts had used a more realistic target for the Marksmanship merit badge, say, for example, A BARN, then I might have earned the Marksmanship merit badge, and with it acquired the confidence to outfit myself in a fashion which would have rebounded to my advantage in the defense of my breakfast.  But, instead, I was left to defend myself with footwear.  Very large footwear, to be sure, but footwear nonetheless.} as well as other applicable outdoor skills.  Desiring the recovery of my copy of the Georgics, and doubting that the beast could have gone far in his condition, I set about equipping myself with the Numerical Essentials (i.e., the items which every prudent person should have in hand before setting out into the woods).
\par
Unfortunately, I've quite lost count how many Numerical Essentials there are.  Twelve?  Six?  Twenty-three?  When I was a youth, the first item was an axe, but when it became de classe to chop down trees, the authorities removed the axe from the list.  To further compound my confusion, it's no longer considered proper, when nature calls, to simply drop, stoop and poop.  Now, a kind of engineering is called for,  and with it new demands for hygiene, all of which requires implements in the form of a little shovel, moist towelettes, and some sort of drying material.  Is that three things, or one thing?  An imponderable, to be sure.  As I considered these matters, I realized that every moment I spent pondering was a moment the beast could use to make good his escape, so I decided the Numerical Essentials were Four: knife, matches, compass, and that which is best referred to as ``the defecation mechanization impedimentia."\footnote{``The defecation mechanization impedimentia" was introduced just after I graduated from the Boy Scouts; however, my younger brother soon reported that he and his fellow Scouts had quickly named it ``the shit kit."  Since then I have learned that Latinate words are preferable in such matters, and are the mark of culture and sophistication.}  Equipping myself accordingly, I set off.
\par
Trailing the beast required that I recall much of what I learned as a youth, although luck was with me in that I was tracking a very large animal which, to judge by the wreckage is left in its path, was making no effort to, as they say, leave no trace.  In fact, its behavior appeared to have been the opposite.  Its trail was marked with churned leaves, broken branches, and the occasional toppled tree.  Tufts of the mysterious fibrous material (fur? feathers?) were caught in brambles along the way, and at spots, particularly where low, dangling vines crossed the trail, the ground appeared as if the beast had been tripped, then struggled to rise.  More good fortune aiding my tracking: I must have, during our battle the night before, injured the beast worse than I thought, for there as a fairly steady and sizable blood trail to follow.
\par
Nevertheless, the beast was, as far as I know, never having seen one before, and thus having no basis for comparison, a stout example of its type, so the search went on far longer that I had projected.  Its trail led over windswept Ozark hills, and down into dark hollows, through briar-choked thickets and across swampy bogs.  I almost lost the trail when it approached the Mississippi, but happily, just as I had stripped naked in preparation for the long swim, a young white boy and and older African-American gentleman came by on a raft, and were quite excited to report that they had seen the beast upstream several miles, crossing back into Missoura.  It seemed that the beast was not yet done, and that it still had the will to effect its escape.
\par
And so upstream I went, after dressing, of course, briars and brambles being what they are, where I picked up the beast's trail again and continued the search.  The sun turned in the sky.  The day became warm, and then, as late afternoon, the temperature moderated.  I began to despair that the light would fail me, and that I would not find the beast before night fell.
\par
Later that evening, just before the sun slipped over the horizon, I followed the trail into a clearing set before an opening in an overhanging cliff.  There, on its back, lay the beast.  In its hands it held two books.  One I recognized as my copy of the Georgics; the other was unfamiliar to me, although from a distance it appeared to be the sort blank diary sold in airport bookstores.  Since the light was slipping away, I approached cautiously, instead of backing off to observe, as I learned to do in the Boy Scouts.
\par
The beast struggled to sit up, but failed, and slumped back to the ground.  Clearly its vital strength was fading fast; in its eyes I detected that dimming light which my daughter's goldfish often got when she forgot to feed them.  I approached closer still.
\par
The beast turned its head—I can't call it a face, since its lineaments were so unfamiliar to me—toward me, and raised one hand in the air.  In the hand its held the books, my copy of Dryden and the airport bookstore diary.  It made as if to speak.
\par
``MO MO," it said, and although it seems to be gasping out its last breaths, its voice was like a roar, so large was its frame.  I approached closer.
\par
``MO MO," it said, weakly, and then, gathering its last breath, ``MO MO GEN MO."  And with that its arm collapsed.  The books fell from its grasp.  And with that, its spirit, whatever sort it was, fled its mortal coil.  The beast was no more; only its lifeless shell remained.
\par
\begin{center}
SUNSET – COYOTES – FIRES – COMFORT IN THE NIGHT
\end{center}
\par
Just as the beast expired—at the exact moment, in fact—the sun slipped below the horizon, leaving only the meager nautical twilight, made even dimmer by the surrounding forest.  Clearly I was going to be spending the night at the beast's lair, since it would have been impossible to retrace my steps in the dark.  I was going to need a fire, and fast; happily I had had the foresight to bring matches, and I had a knife, so I could shave kindling if need be.  So I immediately began to look around for combustible materials.\footnote{Observant readers will note that during my earlier meditations on the Numerical Essentials I had neglected to include a map.  Consequently, even though I was well equipped should, as they say, ``nature call" (and which later proved a prudent provision), etc., I had no map, and thus had no idea where I was.}
\par
Immediately at hand, in the cave under the overhanging cliff which formed the the beast's lair, was an enormous pile of books and notebooks and papers of all sorts of description.  I have read about benighted mountaineers burning their guidebook to stay warm; clearly I would have to sacrifice one to start a fire, although once it was started, I wouldn't suffer from a lack of reading material.
\par
Just then, I was startled by the howls of a pack of coyotes.  They were close, far closer than I preferred.  Like a flash of lightening my situation became clear to me: nightfall was going to force me to camp next to a massive carcass, and with coyotes in the neighborhood.  I was seized by such a dreadful fear and fearful dread that I almost fell to the ground and wept piteously.
\par
But then my gaze returned to the beast's cache of books and papers.  There!  There was my salvation!   I made haste to gather the books and papers and pile them around the beast's corpse, making of them a crematoria with which to dispose of the carcass and discourage the coyotes.  The work went quickly, and just as soon as the light was gone, I was able to light the pyre.  The fire caught, the flames left up, and the slow work of cremating the beast began.  I was saved!\footnote{Note also that among the Numerical Essentials I forgot a source of light.}
\par
The fire crackled merrily, brightly lighting the scene in front of the beast's lair.  And, although the coyotes did come closer, drawn no doubt by the smell of roasting meat, they did not come into the light, but instead stayed in darkness, visible only by the reflection of the fire in their eyes.  Eventually, a strong breeze came up and scattered sparks from the fire and into the forest, which caught fire lustily, and burned with sufficient vigor to drive the coyotes away.
\par
The day had been a long one, and my exertions had been most strenuous, so as a matter of course I grew drowsy, warmed by the flames from the forest and lulled by its cracking and popping.  Eventually I slipped off to sleep.
\par
At some point in the night, I awoke to the sensation of having my forehead caressed by something warm and soft.  I opened my eyes, and my eyes fell upon a creature not unlike that of the beast.  But instead of the beast's brutish heft, this creature had a softer, more gentle aspect, its visage lined with what was clearly an expression of both sorrow and concern, and of a need for comfort and to comfort.  The creature and I fell into each others' tender embrace, and such events ensued that even I, a man of the world, blush to recall them.  After a while, further exhausted by these additional exertions, pleasant though they certainly were, I fell into a deep slumber, and passed the rest of the night dreamlessly.
\par
\begin{center}
MORNING AGAIN – AN IMPROVISED, BUT HEARTY AND HEALTHY, BREAKFAST – ASTONISHMENT – ZOOLOGICAL RECLASSIFICATION
\end{center}
\par
The sun rose, as it does every morning, and with it, me, a process which, although regular, will not prove perpetual, the fact which ought rightly to compel us all to self-examination and repentance.  In any case, I awoke as I always do: famished, and doubly so by the exertions of the previous day and night.  But I seemed to be luckless, for I had packed no breakfast in my pack, and so faced the prospect of want and deprivation.
\par
Searching around the beast's lair, surrounded by the dying embers of the forest fire, I noticed that the crematoria I had prepared for the beast had not completely consumed its body.  One of its two large yellow feet had fallen out of the fire, and instead of being consumed like the rest of its carcass, had been roasted to a prefect, crispy doneness.  Giving thanks for this fortuitous provision, I tucked into my improvised breakfast with a relish, the meat having a taste not unlike that of chicken.
\par
As I breakfasted, I notice lying on the ground the two books the beast had held in its dying grasp, and which somehow had been overlooked when I gathered materials for the pyre.  One book was my copy the Georgics, which I set aside, since I knew how it ends.  The other, the airport bookstore blank book, I took in hand, and looked it over as I broke my fast.
\par
The beast's large yellow foot was quite large, and it took me some time to get through it, which meant that I was able to examine quite a bit of the blank book, which was no longer blank, but had been written in throughout in a legible but odd hand.  The subject seemed to be in some way a variation on the Georgics; not an exact transcription exactly, but rather one with certain consistent substitutions.  There seemed to be some sort of art to it, and it looked familiar, but I couldn't quite put my finger on it.
\par
The events of the past nights and day had been most stressful, and I wasn't congitating clearly, although something seemed seemed more than odd.  Disjointed sounds swirled through my recollection: the beast's regular, if not comprehensible, utterances when it invaded my camp, and its final sounds as it expired.  ``MO MO", it had said, and then repeated the syllables with its last breaths.
\par
As I chewed on one of the monster's two large yellow feet, the only remaining one, which was growing smaller by the bite, the conclusion hit me like a bolt of lightening.  ``Dammit," I cried aloud, ``that was no beast!  That was Bigfoot!"\footnote{Some persons might question my unwillingness to definitively state the gender of either Monster or the gentler creature.  I, too, not surprisingly given these events, have tendered the question much thought and reflection.  I now believe that the fibrous hair- or feather-like material which I have observed in my camp and along Monster's trail, and which made so soft and warm the touch of the gentler creature, was a material similar to that found in the plumes of the maribou.  My tentative conclusion is that Bigfoot is not a mammal, but an enormous, flightless bird, its form not unlike that of a gigantic penguin.  That many observers mistake it for a large hominid is not unexpected, since the appearances of one of Monster's kind are always attended by such fear and confusion as would be sufficient to cloud the senses of even the most resolute of observers.  In any case, my conjecture is that determining the sex of any particular Bigfoot purely by visual inspection is impossible, as it often is with flightless avians.}
\par
The clue was in the monster's last breath: ``MO MO," it called itself, short for ``Missoura Monster", the megapedaled Ozarkian cousin to the Cascadian \textit{sásq'ets}.  For years unsubstantiated rumors of Mo-Mo had circulated through the state, and there I found myself, deep on the edge of the Ozarks, holding in my hand evidence that Mo-Mo existed, that Mo-Mo was real!
\par
I continued my breakfast, and when I was done, and had tossed the bones onto a pile of embers from the forest fire, I paused to consider the full import of Monster's words.  ``MO MO GEN MO," it said.  In an instant I realized where I had seen substitutions like those in the airport bookstore blank book: the reminded me of the results of certain Continental schemes for generating literature.  Monster, with its last words, was trying to say to me that it was engaged in a kind of avant garde literary production!
\par
But by this time, however, as is common after such a large breakfast, the gastrocolic reflex was going to have its way with me, so I set off in search of my bag of impedimentia.
\par
\begin{center}
A LAST REALIZATION – WILLIAM CLARK'S ORTHOGRAPHICAL DIFFICULTIES – A FURTHER SEARCH – RETURN TO CIVILIZATION – A THIRD HEALTHY AND HEARTY BREAKFAST
\end{center}
\par
When I had returned, I turned my thoughts to Monster's last words.\footnote{I mean no disrespect by continuing to call the monster ``Monster," that being its name, and apparently the name it chose for itself.  I merely refer to it as I would any other creative figure, by its last name.  I wouldn't presume to call it ``Mo-Mo", which seems disrespectful and overly familiar, as it would be to refer to Shakespeare as ``Shaky" or to Milton as ``Milty."}  What was the meaning of the last syllable?  ``Missoura Monster Generates Monster?"  That, it seemed at the time, was an obvious fact, scarcely worth uttering with one's dying breath.  ``Missoura Monster Generates Missouruh?"  That didn't seem right: everyone, or at least every true son of Missoura, knows that William Clark is the father of Missoura.\footnote{The word is spelled correctly, since it reflects the manner in which the early settlers of our state pronounced its name.  The modern spelling, ``Missouri," is erroneous, and is believed to have originated with William Clark, who, despite his many and numerous virtues, both civic and manly, was handicapped by a haphazard and erratic command of English orthography, as any reader of his journals, among who are included any reasonably educated Missouran, is sure to understand.}  A mostly consumed page or two from the remains of the pyre blew past my foot, and it dawned on me: ``More!" Monster had meant ``more"!
\par
There didn't seem to be a whole lot of ``more" left; most had gone into firing the monster's pyre, and by a happy accident into firing the forest.  But still, it seemed prudent—for I am nothing but prudent—to search through the cave beneath the overhanging cliff, which I promptly did, and found, tucked into a corner and thus overlooked in my search on the previous evening for combustible materials, a cache of notebooks and papers of various sizes and shapes, all filled with the same clear but curious hand.
\par
Providence continued to smile upon me, for among the cache was a map of the area.  Although the back of the map was covered with the monster's handwriting, the front was legible.  Making use of it, and navigating by means of my compass, I located a nearby road, where I happened across a crowd of firefighters, brought in to quell the forest fire.  One of them drove me back to camp where, reunited with my very large and very red cooler, I sat down to a hearty and healthy late breakfast and, safe and full at last, I fell into a deep sleep.
\newpage
\par
\begin{center}
PHILOLOGICAL INQUIRIES – FINDINGS – BASIC FACTS REGARDING THE FRENCH IN MISSOURA – CONCLUSIONS
\end{center}
\par
Upon returning back to St. Louis, I set about examining the cache of documents.  Unfortunately, the only original text which survived was my copy of the Georgics, which, along with Monster's generative version, provided me with a means of understanding Monster's substitutionary scheme, and of eventually identifying the source texts for its literary products.  This required transcribing the documents, which form the basis for THE MOMOGENMO TREASURY OF N-PLUS LITERATURE.
\par
Using various text-mining techniques, which I have perfected while working in the Humanities Digital Workshop at Washington University in St. Louis, I was eventually able to identify the source text (i.e., the books consumed in Monster's pyre) for a few more than 120 of the transcriptions.  These transcriptions are mounted on THE TREASURY'S digital archive; the remainder of Monster's productions, numbering about 550, are in my safekeeping.
\par
Monster's generative method seems to have been a variation on the N+7 scheme supposedly pioneered by the French so-called ``savants" of the \textit{Ouvroir de littérature potentielle}, except that instead of using a dictionary to identify substitutions, Monster seems to have had some notion of related words, similar to what we would derive computationally from word vectors.  Since I saw no evidence of any sort of such equipment at Monster's lair, I can only assume that it possessed a word-vector-like mental model of several languages (English, Latin, and Greek; and a case can be made for Monster having at least a passing familiarity with other Continental vernacular literatures).  To obtain and use such a mental model is of course a breathtaking intellectual achievement.  Now that Mr. Dillon, whose enunciation is scarcely better than Monster's, and who stands as a pygmy next to Monster's towering intellect, has been awarded the Nobel prize, I can only look forward to the day when the Nobel Committee recognizes Monster's genius.
\par
Using Monster's own method, I was able to generate the contents of the Descriptive Catalog which follows.  I considered describing Monster's artistic output in my own words; however, it seemed fitting that Monster's method speak for Monster's work.  The attached Bibliography, which follows the Descriptive Catalog, lists the texts in THE TREASURY and accounts for their sources.
\par
I am fairly certain about the strands of influence between Monster and the so-called ``savants" of the \textit{Ouvroir de littérature potentielle}.  It would be slovenly to suppose that the influence flowed from France to the area surrounding Monster's lair.  True, at one time the area was French, and a notable French town (St. Genevieve) lies not far from Monster's lair; however, French influence declined rapidly after the Louisiana Purchase, and there has been no transfer of new French culture to the region since.
\par
Instead, it seems almost certain that Monster's method moved downstream from St. Genevieve, through the Creole \textit{entrepôt} of New Orleans, and from there to France, the way everything else of value in Middle America flows to Europe.  And there is evidence also in the changes in the n-plus method as it moved from Monster to \textit{Ouvroir de littérature potentielle}, where it was practiced in a simpler and debased form centered on a language model much more reductive than the one employed by Monster. 
\par
Regardless of their opinions about forest fires, many readers will find this a bittersweet story: Monster dead and burned, along with most of its books and papers.  And the part of Monster not consumed by the flames, eaten for breakfast.  But such readers may take solace in THE MOMOGENMO TREASURY OF N-PLUS LITERATURE, an enduring monument to Monster's genius, and in the thought that, somewhere in the Ozarks, another like Monster lives, a gentler creature who, if our assignation proved fruitful, may have already brought forth new forms.
\par
\textit{Vive le monstre!}
